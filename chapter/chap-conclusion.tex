\chapter{总结}
\label{chap:conclusion}

在很多实际应用中,数据的类别之间存在一个自然的序关系。处理这样一类有序数据的问题称之为序回归问题。序回归广泛应用在情感分析、信息检索、推荐系统、心理学、金融、医学等领域。作为机器学习里一个有代表性的问题,序回归和传统的分类、回归问题都具有一定的相似性。早期的研究往往忽略数据中的序关系,将其当作传统的标称分类问题。然而,序关系作为有序数据中重要的先验知识,忽略它将对结果产生负面影响。从2000年左右开始,序回归问题逐渐受到研究者们的关注,同时涌现出了很多专门针对有序数据的序回归技术。已有的研究主要集中在处理有监督序回归问题,即只使用有标签数据去训练模型。对有监督序回归问题的研究较深,并出现了多种有监督序回归技术。然而,有监督序回归技术的缺陷之一在于,它需要足够的有标签数据来训练模型。在很多实际应用中,有标签数据往往难以获得,并且校对起来代价很高。而无标签数据通常大量存在,并且易于获取。因此,同时考虑有标签数据和无标签数据的半监督序回归问题具有重要的研究意义和实际价值。本文主要以此为切入点,对半监督序回归问题做了一定的研究和探索。

本文首先在第二章回顾了序回归技术的发展历程,总结序回归的发展趋势及不足之处,便于我们展开后续的研究。我们分别从有监督序回归技术、半监督序回归技术和基于演化算法的序回归技术三个方向介绍了一些有代表性的算法,并根据方法特点进行了梳理和分类。常见的有监督序回归技术主要可以分成三类方法。第一类方法将序回归问题当作传统的标称分类问题或者传统的回归问题,使用传统的分类或回归方法来处理。这类方法的主要问题是在没有先验知识的情况下,很难对标签和标签之间的差别进行准确地度量。第二类方法先通过一定的分解策略,将序回归问题分解成多个二分类子问题,再使用传统的分类模型来处理每个子问题,最后对子问题的结果进行合并得到原问题的结果。通过设计分解策略,可以数据中的序 信息保留下来。因此,相比较于第一类方法,这类方法能够比较有效地利用序信息,在性能上有所提升。但难点也显而易见,即如何设计好的分解策略和好的合并策略。第三类方法通过拓展传统的分类模型来使序信息加入模型中进行训练,例如显式增加一个序关系约束条件。这种方法相比较于前两种方法,对序 信息的处理更加直接,通常效果也更好。第三类方法是有监督序回归技术中主流的一类方法,其中比较有代表性的算法有SVOR、KDLOR、GPOR等。对于有监督序回归问题的研究较为深入,而对半监督序回归问题的研究尚浅。由于序回归问题特点,传统的半监督方法难以直接应用到半监督序回归问题中。本文认为,研究特定的半监督序回归技术具有重要价值,这也是本文研究的主要动机。此外,本文还介绍了利用演化算法来处理序回归问题的一 些相关工作。在处理序回归问题时,演化算法会在传统优化算法难以处理的问题上有所帮助。

第三章提出了一种半监督序回归技术——基于加权核判别分析的半监督序回归算法(WKFDOR)。WKFDOR算法是基于KDLOR算法做的半监督拓展。KDLOR通过序约束来使用序信息,属于第二章中提到的第三类有监督序回归技术。KDLOR算法思想是,找到一个最佳投影向量,该投影向量可以使相邻序的类别之间有大的间隔,而每个类别内部的数据能够有小的方差,并同时保证序的正确性。由于KDLOR是基于FDA算法的,它能够充分利用数据分布信息。基于支持向量机的序回归算法是通过支持向量来决定分类面的,所有它可能造成找到的投影向量不合理。此外,相比较于一些主流的有监督序回归算法,KDLOR有相对较低的计算复杂度,同时还有不错的预测性能。考虑到KDLOR在有标签数据不足时,难以准确估计类的中心趋势。因此,WKFDOR算法的主要切入点是通过同时使用有标签数据和无标签数据来更准确地估计类分布。其算法步骤主要包括:计算无标签数据对于每个类别的隶属度;将隶属度作为权重,使用一种加权策略来计算在整个训练集上的类分布;将加权类分布应用到KDLOR框架得到模型,并求出最优投影向量。我们分别在合成数据集和10个真实数据集上来验证WKFDOR的有效性。通过可视化合成数据集的结果,验证了WKFDOR能够有效利用无标签数据估计类分布,从而提升了泛化性能。在真实数据集上的结果验证了WKFDOR算法能够在大多数数据集上获得更优的性能,尤其是MZE指标。我们分析了WKFDOR在一些数据集上的MAE结果没有体现优越性的原因,可能是WKFDOR中用于估计隶属度的算法没有考虑序信息,导致对权重的估计不是很准确。

针对第三章中WKFDOR算法的问题,本文在第四章提出了改进的算法——ESSOR。ESSOR的算法思想是优化初始权重,以得到更优的学习性能和泛化能力。由于将权重作为优化变量,并以学习性能和泛化能力作为优化目标,所以这是一个非凸、不可导的优化问题。传统的优化算法难以处理,因此我们采用演化算法。本文提出了一种有效的个体表示方法,能够将问题维度从\((N-L)\times K\)降到\(K\)。另外,本文使用了一种组合型的适应度函数,从而兼顾模型的学习性能和泛化能力。在具体算法实现中,我们采用差分进化算法来求解这个连续优化问题。我们通过在第三章中提到的10个真实数据集上对比KDLOR、WKFDOR和ESSOR算法,验证了ESSOR能够更有效地利用无标签数据来提升性能。此外,相比较于WKFDOR,ESSOR体现出了在不同数据集上的稳定性。本文还提出了一种基于随机采样的加速方法(ESSOR-S)来提升算法的运行速度,用于处理大规模数据集。对于Sushi和Connect-4数据集,我们使用ESSOR-S来优化权重。实验结果显示,ESSOR-S依然能够有效提升序回归效果。

据我们所知,ESSOR是第一个基于演化算法的半监督序回归技术。在处理半监督序回归问题时,演化算法不失为一种好的优化工具。在后续的工作中,我们研究如何更高效地评估适应度和更准确地使用无标签数据。关于序回归问题,还有很多值得研究的地方。下面我们提出几个有意思的研究方向:
\begin{enumerate}
\item[1.]目前最常用的序回归性能指标是MAE和MZE,研究更能体现序回归问题特点的性能指标将具有指导性作用。
\item[2.]序回归和偏好学习(preference learning)、排序学习(learning to rank,L2R)有很强的相似性,如何借助彼此之间的关系来取长补短是一个值得研究的问题。
\item[3.]在很多实际应用中,数据以流的形式到来,研究增量序回归技术将具有重要的实际应用价值。
\item[4.]近年来,深度学习(deep learning,DL)受到学术界和工业界的追捧。随着计算资源的提升,深度学习在一些机器学习问题上凸显出了一定的优越性。能否将深度学习应用到有序数据中,更深层次地挖掘数据中的序关系,也是一个有趣的研究方向。
\end{enumerate}


